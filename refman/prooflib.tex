\chapter{Developping Mathematics with Prooflib}
\label{chapt:prooflib}
Lisa's kernel offers all the necessary tools to develops proofs, but  reading and writing proofs written directly in its language is cumbersome.
To develop and maintain a library of mathematical development, Lisa offers a dedicate interface and DSL to write proofs: Prooflib
Lisa provides a canonical way of writing and organizing Kernel proofs by mean of a set of utilities and a DSL made possible by some of Scala 3's features.
\Cref{fig:theoryFileExample} is a reminder from \Cref{chapt:quickguide} of the canonical way to write a theory file in Lisa.

\begin{figure}
  \lisaCode{src/MyTheoryName.scala}{An example of a theory file in Lisa\label{fig:theoryFileExample}}{firstline=3}
\end{figure}

In this chapter, we will describe how each of these construct is made possible and how they translate to statements in the Kernel.

\section{Richer FOL}

The syntax of Prooflib is similar to the syntax of Lisa's kernel, but the \textit{Sorts}, such as Ind and Prop, are reflected in Scala's type system, making well-sortedness checked at compile time and offering more detailed documentation and features. Prooflib's syntax also supports custom printing, such as infix notation, special handling for binders, and more.

\subsection{Sorts and Expressions}

\begin{definition}[Sorts]\phantom{.}
  \begin{lstlisting}[language=scala]
trait Ind
trait Prop
infix trait >>:[I, O]
  \end{lstlisting}

\end{definition}

\begin{definition}[Expressions]
  Expressions in Prooflib always correspond to an underlying expression in lisa's kernel, which can bee accessed using \lstinline|myExpr.underlying|. Expressions are always \textit{sorted}, and this sort reflects in their scala type.
  \begin{lstlisting}[language=scala]
trait Expr[S]
case class Variables[S](name: String) extends Expr[S]
case class Constants[S](name: String) extends Expr[S]
case class App[S, T](f: Expr[S >>: T], arg: Expr[S]) extends Expr[T]
case class Abs[S, T](v: Variable[S], body: Expr[T]) extends Expr[S >>: T]
  \end{lstlisting}
\end{definition}
\noindent
Expressions are usually built with the following helpers:
\begin{example}\phantom{.}

  \begin{lstlisting}[language=scala]
val x = variable[Ind] //the name "x" is used automatically
val c = variable[Ind]
val ∈ = constant[Ind >>: Ind >>: Prop]
val f = variable[(Ind >>: Prop) >>: Ind]

x ∈ c : Expr[Prop]
lambda(x, x ∈ c) : Expr[Ind >>: Prop]
f(lambda(x, x ∈ c)) : Expr[Ind]
  \end{lstlisting}

\end{example}

Expressions also support substitutions.

\begin{definition}[Substitution]\phantom{.}
  Substitution are most often performed with \lstinline|SubstPair|s, which guarantee well-sortedness.
  \begin{lstlisting}[language=scala]
trait SubstPair extends Product:
  type S
  val _1: Variable[S]
  val _2: Expr[S]

(x := f(∅)) : SubstPair
g(x, y).subst(x := f(∅), y := x) // == g(f(∅), x)
  \end{lstlisting}
  \noindent
  but can also be performed unsafely, when sorts are not necessarily known:
  \begin{lstlisting}[language=scala]
//if ill-sorted, may crash in unpredictable ways.
myExpr.substituteUnsafe(Map(x -> s, y -> t))

//with sanity runtime check for well-sortedness
myExpr.substituteWithCheck(Map(x -> s, y -> t))
  \end{lstlisting}
\end{definition}

\subsection{Sequents}

Expressions build into sequents, which again have an underlying sequent in the kernel.

\begin{definition}[Sequents]
  Sequents are formally pairs of sets of \lstinline|Expr[Prop]|.
  \begin{lstlisting}[language=scala]
case class Sequent(left: Set[Expr[Prop]], right: Set[Expr[Prop]])
  \end{lstlisting}
  \noindent
  Sequent can be built from formulas and collections of formulas:
  \begin{lstlisting}[language=scala]
val s1 = (x ∈ c) |- (f(x) ∈ f(c))
val s2 = () |- (f(x) ∈ f(c))
val s3 = (x ∈ c) |- ()
val s4 = Set(x ∈ c, y ∈ c) |- Set(x = y, x \in y)
val s4 = assumptions |- (x = c)
  \end{lstlisting}
  the logical semantics of sequents is the same as in the kernel, i.e. a sequent is valid if and only if the conjunction of its left side implies the disjunction of its right side.
  But it is usually discouraged to have multiple formulas on the right side of a sequent in theorems and lemmas, as it is harder to understand. Using multiple formulas on the right side of a sequent is however allowed in intermediate steps of a proof and in proof tactics.

  Sequents, like expressions, support substitutions:
  \begin{lstlisting}[language=scala]
val s = Sequent(Set(x ∈ c), Set(f(x) ∈ f(c)))
s.substitute(x := g(∅))
  // == Sequent(Set(g(∅) ∈ c), Set(f(g(∅)) ∈ f(c)))
  \end{lstlisting}
\end{definition}

\section{Proof Builders}

\subsection{Proofs}

\subsection{Facts}

\subsection{Instantiations}

\subsection{Local Definitions}
\label{sec:localDefinitions}
The following line of reasoning is standard in mathematical proofs. Suppose we have already proven the following fact:
$$∃ x. P(x)$$
And want to prove the property $\phi$.
A proof of $\phi$ using the previous theorem would naturally be obtained the following way:
\begin{quotation}
  Since we have proven $∃ x. P(x)$, let $c$ be an arbitrary value such that $P(c)$ holds.
  Hence we prove $\phi$, using the fact that $P(c)$: (...).
\end{quotation}
However, introducing a definition locally corresponding to a statement of the form
$$∃ x. P(x)$$
is not a built-in feature of first-order logic.  This can however be simulated by introducing a fresh variable symbol $c$, that must stay fresh in the rest of the proof, and the assumption $P(c)$. The rest of the proof is then carried out under this assumption. When the proof is finished, the end statement should not contain $c$ free as it is a \textit{local} definition, and the assumption can be eliminated using the LeftExists and Cut rules. Such a $c$ is called a \textit{witness}.
Formally, the proof in (...) is a proof of $P(c) ⊢ \phi$. This can be transformed into a proof of $\phi$ by mean of the following steps:
\begin{center}
  \AxiomC{$P(c) ⊢ \phi$}
  \UnaryInfC{$\exists x. P(x) ⊢  \phi$}
  \RightLabel{\text { LeftExists}}
  \AxiomC{$\exists x. P(x)$}
  \RightLabel{\text { Cut}}
  \BinaryInfC{$\phi$}
\end{center}
Not that for this step to be correct, $c$ must not be free in $\phi$. This correspond to the fact that $c$ is an arbitrary free symbol.

This simulation is provided by Lisa through the \lstinline|witness|{} method. It takes as argument a fact showing $∃ x. P(x)$, and introduce a new symbol with the desired property. For an example, see \Cref{fig:localDefinitionExample}.

\begin{figure}
  \begin{lstlisting}[language=lisa, frame=single]
    val existentialAxiom = Axiom(exists(x, in(x, emptySet)))
    val falso = Theorem( ⊥ ) {
      val c = witness(existentialAxiom)
      have( ⊥ ) by Tautology.from(
            c.definition, emptySetAxiom of (x := c))
    }
  \end{lstlisting}
  \caption{An example use of local definitions in Lisa}
  \label{fig:localDefinitionExample}
\end{figure}

\section{DSL}

\subsection{Instantiations with ``of''}

With lisa's kernel, it is possible to instantiate a theorem proving $P(x)$ to obtain a proof of $P(t)$, for any term $t$, using the \texttt{Inst} rule from \Cref{fig:deduct_rules_1}. Lisa's DSL provides a more convenient way to do so, using the \lstinline|of| keyword. It is used like so:
\begin{lstlisting}[language=lisa, frame=single]
  val ax = Axiom(P(x))
  val falso = Theorem(P(c) /\ P(d)) {
    have(thesis) by RightAnd(ax of (x := c), ax of (x := d))
  }
\end{lstlisting}
\lstinline|x := d| is called a substitution pair, and is equivalent to the tuple \lstinline|(x, d)|. Arbitrarily many substitution pairs can be given as argument to \lstinline|of|, and the instantiations are done simultaneously. \lstinline|ax of (x := c)| is called an \lstinline|InstantiatedFact|, whose statement is $P(c)$, and which can be used exactly like theorems, axioms and intermediate steps in the proof. Internally, Lisa produces a proof step corresponding to the instantiation using the \texttt{Inst} rule.

The \lstinline|of| keyword can also instantiate universally quantified formulas of a fact, when it contains a single formula. For example, the following code is valid:
\begin{lstlisting}[language=lisa, frame=single]
  val ax = Axiom(∀(x, P(x)))
  val thm = Theorem(P(c) /\ P(d)) {
    have(thesis) by RightAnd(ax of c, ax of d)
  }
\end{lstlisting}
Here, \lstinline|ax of c| is a fact whose proven statement is again $P(c)$. It is possible to instantiate multiple $\forall$ quantifiers at once. For example if \lstinline|ax| is an axiom of the form $∀ x, ∀ y, P(x, y)$, then \lstinline|ax of (c, d)| is a fact whose proven statement is $P(c, d)$. It is also possible to combine instantiation of free symbols and quantified variables. For example, if \lstinline|ax| is an axiom of the form $∀ x, ∀ y, P(x, y)$, then \lstinline|ax of (c, y, P := ===)| is a fact whose proven statement is $(c = y)$.

Formally, the \lstinline|of| keyword takes as argument arbitrarily many terms and substitution pairs. If there is at least one term given as argument, the base fact must have a single universally quantified formula on the right (an arbitrarily many formulas on the left). The number of given terms must be at most the number of leading universal quantifiers. Moreover, a substitution cannot instantiate any locked symbol (i.e. a symbol part of an assumption or definition). The ordering of substitution pairs does not matter, but the ordering of terms does. The resulting fact is obtained by first replacing the free symbols in the formula by the given substitution pairs, and then instantiating the quantified variables in the formula by the given terms

In general, for the following proof
\begin{lstlisting}[language=lisa, frame=single]
  val ax = Axiom(∀(x, ∀(y, P(x, y))))
  val thm = Theorem(c == d) {
    have(thesis) by Restate.from(ax of (c, d, P := ===))
  }
\end{lstlisting}
Lisa will produce the following inner statements:
\begin{lstlisting}
   -1 Import 0                 (  ) ⊢  ∀(x, ∀(y, P(x, y)))
    0 SequentInstantiationRule (  ) ⊢  c === d
\end{lstlisting}
